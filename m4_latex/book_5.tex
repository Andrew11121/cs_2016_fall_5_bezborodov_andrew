\documentclass[a4paper]{article}

%% Language and font encodings
\usepackage[english]{babel}
\usepackage[utf8x]{inputenc}
\usepackage[T1]{fontenc}

%% Sets page size and margins
\usepackage[a4paper,top=3cm,bottom=2cm,left=3cm,right=3cm,marginparwidth=1.75cm]{geometry}

%% Useful packages
\usepackage{amsmath}
\usepackage{graphicx}
\usepackage[colorinlistoftodos]{todonotes}
\usepackage[colorlinks=true, allcolors=blue]{hyperref}

\title{Your Paper}
\author{You}

\begin{document}
2
Easy cases
2.1 Gaussian integral revisited 13
2.2 Plane geometry: The area of an ellipse 16
2.3 Solid geometry: The volume of a truncated pyramid 17
2.4 Fluid mechanics: Drag 21
2.5 Summary and further problems 29
A correct solution works in all cases, including the easy ones. This maxim
underlies the second tool—the method of easy cases. It will help us guess
integrals, deduce volumes, and solve exacting differential equations.
2.1 Gaussian integral revisited
As the first application, let’s revisit the Gaussian integral from Section 1.3,
∞
−∞
dx. (2.1)
The correct choice must work for all a  0. At this range’s endpoints
(a = ∞ and a = 0), the integral is easy to evaluate.
What is the integral when a = ∞?
e−10x2
0 1
As the first easy case, increase a to ∞. Then −ax2 becomes
very negative, even when x is tiny. The exponential
of a large negative number is tiny, so the bell curve
narrows to a sliver, and its area shrinks to zero. Therefore,
as a → ∞ the integral shrinks to zero. This result refutes the option
14 2 Easy cases
which is zero when a = ∞.
What is the integral when a = 0?
e−x2/10
0 1
In the a = 0 extreme, the bell curve flattens into a
horizontal line with unit height. Its area, integrated
over the infinite range, is infinite. This result refutes
the a option, which is zero when a = 0; and it
passes both easy-cases tests.
If these two options were the only options, we would choose a. However,
if a third option were

2/a, how could you decide between it and
a ? Both options pass both easy-cases tests; they also have identical
dimensions. The choice looks difficult.
To choose, try a third easy case: a = 1. Then the integral simplifies to
∞
−∞
e−x2
dx. (2.2)
This classic integral can be evaluated in closed form by using polar coordinates,
but that method also requires a trick with few other applications
(textbooks on multivariable calculus give the gory details). A less elegant
but more general approach is to evaluate the integral numerically and to
use the approximate value to guess the closed form.
Therefore, replace the smooth curve e−x2
with a curve
having n line segments. This piecewise-linear approximation
turns the area into a sum of n trapezoids. As
n approaches infinity, the area of the trapezoids more and more closely
approaches the area under the smooth curve.
n Area
10 2.07326300569564
20 1.77263720482665
30 1.77245385170978
40 1.77245385090552
50 1.77245385090552
The table gives the area under the curve in the
range x = −10 . . . 10, after dividing the curve
into n line segments. The areas settle onto a
stable value, and it looks familiar. It begins
with 1.7, which might arise from 
3. However,
it continues as 1.77, which is too large to be 
3.
Fortunately,  is slightly larger than 3, so the
area might be converging to .
2.1 Gaussian integral revisited 15
Let’s check by comparing the squared area against:
1.772453850905522 ≈ 3.14159265358980,
pi ≈ 3.14159265358979. (2.3)
The close match suggests that the a = 1 Gaussian integral is indeed :

Easy cases are not the only way to judge these choices. Dimensional analysis,
for example, can also restrict the possibilities (Section 1.3). It even
eliminates choices like √pi/a that pass all three easy-cases tests. However,
easy cases are, by design, simple. They do not require us to invent or
deduce dimensions for x, a, and dx (the extensive analysis of Section 1.3).
Easy cases, unlike dimensional analysis, can also eliminate choices like
2/a with correct dimensions. Each tool has its strengths.
Problem 2.1 Testing several alternatives
For the Gaussian integral

use the three easy-cases tests to evaluate the following candidates for its value.
(a) √pi/a (b) 1 + (√pi − 1)/a (c) 1/a2 + (√pi − 1)/a.
Problem 2.2 Plausible, incorrect alternative
Is there an alternative to pi/a that has valid dimensions and passes the three
easy-cases tests?
16 2 Easy cases
Problem 2.3 Guessing a closed form
Use a change of variable to show that

1 + x2 . (2.8)
The second integral has a finite integration range, so it is easier than the first
integral to evaluate numerically. Estimate the second integral using the trapezoid
approximation and a computer or programmable calculator. Then guess a closed
form for the first integral.
2.2 Plane geometry: The area of an ellipse

The second application of easy cases is from plane
geometry: the area of an ellipse. This ellipse has
semimajor axis a and semiminor axis b. For its area A
consider the following candidates:
(a) ab2 (b) a2 + b2 (c) a3
/b (d) 2ab (e) piab.
What are the merits or drawbacks of each candidate?
The candidate A = ab2 has dimensions of L3, whereas an area must have
dimensions of L2. Thus ab2 must be wrong.
The candidate A = a2 + b2 has correct dimensions (as do the remaining
candidates), so the next tests are the easy cases of the radii a and b. For a,
the low extreme a = 0 produces an infinitesimally thin ellipse with zero
area. However, when a = 0 the candidate A = a2 + b2 reduces to A = b2
rather than to 0; so a2 + b2 fails the a = 0 test.
The candidate A = a3/b correctly predicts zero area when a = 0. Because
a = 0 was a useful easy case, and the axis labels a and b are almost
interchangeable, its symmetric counterpart b = 0 should also be a useful
easy case. It too produces an infinitesimally thin ellipse with zero area;
alas, the candidate a3/b predicts an infinite area, so it fails the b = 0 test.
Two candidates remain.
The candidate A = 2ab shows promise. When a = 0 or b = 0, the
actual and predicted areas are zero, so A = 2ab passes both easy-cases
tests. Further testing requires the third easy case: a = b. Then the ellipse
becomes a circle with radius a and area pia2. The candidate 2ab, however,
reduces to A = 2a2, so it fails the a = b test.
\end{document}
